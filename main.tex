\documentclass[a4paper,dvipsnames,11pt,dvipdfmx,backend=luatex]{amsart} %lualatex
\usepackage{preamble}

\begin{document}
\maketitle
\section{導入}
	Simplicial  setの議論をするとき、ある一つのsimplicial setのsimplexたちを議論するのに人々はしばしばダイアグラムを使います。
	このダイアグラムは正確にはsimplicial setのダイアグラムというよりはそのnon-degenerate cellたちからなるsemi-simplicial setのダイアグラムです。
	それでもそのようなダイアグラムを用いた議論が有用であるという事実は、ある意味でsimplicial setはnon-degenerateなcellたちだけから構成されている、ということを示唆しています。
	このことは、厳密には次の事実によって確かめられます:任意のsimplicial setはnon-degenerateなsimplexでindexされたboundary inclusionのcolimitで表示できる(skeletal filtration).

	一方で、simplicial setの圏$\sSet$はpresheaf categoryなので、任意のsimplicial set $X$はsimplexのcolimitをつかって$X\cong\colim(\Simp/X\to\Simp\to\sSet)$とかけます。
	ここでcategory of elements $\Simp/X$は$X$のcellのなす圏です(Yoneda lemma)。
	上の観察から、「このcolimitは$\Simp/X$のnon-degenerate cellのなすfull subcategoryに制限できるか?」という問いが自然であることがわかるでしょう。
	実はこれは一般には成立しません。
	例えば、一つの$0$-simplexとその上のdegenerate 1-cellをboundaryにもつ一つの$2$-simplexからなるsimplicial set $X$を考えると
	$\colim(\ndcat[\Simp]{X}\to\Simp/X\to\Simp\to\sSet)$はnon-degenerateな$1$-cellを持つことになり、もとの$X$と異なるものになります。

	この記事では、このcolimitが$X$を復元するための十分条件である「non-degenerate cellたちが$\Simp/X$のfinal full subcategoryをなす」という条件が
	fondind(Face Of Non-Degenerate Is Non-Degenerate)とここで呼ぶことにする条件と同値になることを、任意のelegant Reedy categoryに$\Simp$を一般化した上で示します。
	このことはfondindなsimplicial setがある強い意味でnon-degenerateなcellたちだけからなっていることを意味します。
	さらにこのクラスは(many-sorted) quasivarietyをなすことがわかります。
\section{Preliminary (elegant Reedy category)}
This section is mainly based on \cite[Section 3]{BR13}.
\begin{definition}
	A \emph{Reedy category} $\dgm{A}$ consists of the following data:
	\begin{itemize}
		\item %
			A small category $\dgm{A}$.
		\item %
			A strict factorisation system $(\dgm{A}_-,\dgm{A}_+)$.
			We consider $\dgm{A}_-$ and $\dgm{A}_+$ as wide subcategories of $\dgm{A}$.
		\item %
			A function $d\colon\Obj\dgm{A}\arr\omega$
			that extends to identity-reflecting functors
			$d\colon\dgm{A}^\op_-\arr\omega$ and $d\colon\dgm{A}_+\arr\omega$,
			where the set of natural numbers $\omega$ is seen as a linearly ordered set.
		\qedhere %
	\end{itemize}
\end{definition}
\begin{notation}
	Let $\dgm{A}$ be a Reedy category and $X\in\Psh[\dgm{A}]$ be a presheaf.
	We view the category $\dgm{A}$ as a full subcategory of $\Psh[\dgm{A}]$.
	\begin{itemize}
		\item %
			We mean by $\arr[two heads]$ a morphism in $\dbl{A}_-$,
			while by $\arr[tail]$, we mean a morphism in $\dbl{A}_+$.
		\item %
			A \emph{cell} is a morphism $a\arr X$ in $\Psh[\dgm{A}]$ whose domain $a$ is in $\dgm{A}$.
		\item %
			A cell $x\colon a\arr X$ is \emph{degenerate} if there is a factorisation $a\arr"\sigma"[two heads]b\arr"y"X$ of $x$
			such that $\sigma$ is non-identity morphism in $\dgm{A}_-$.
		\item %
			For each cell $x\colon a\arr X$,
			a \emph{face} of $x$ is a cell $y\colon b \arr X$ that factors as $b\arr"\iota"[tail]a\arr"x"X$ with $\iota$ in $\dgm{A}_+$,
			while a \emph{degeneracy} of $x$ is a cell $y\colon b \arr X$ that factors as $b\arr"\sigma"[two heads]a\arr"x"X$ with $\sigma$ in $\dgm{A}_-$.
		\qedhere %
	\end{itemize}
\end{notation}
\begin{lemma}
	\label{lem:existenceofNon-deg_factorisation}
	Let $\dgm{A}$ be a Reedy category.
	For each presheaf $X\in\Psh[\dgm{A}]$ and any cell $x\colon a\arr X$, there exits a factorisation
	$a\arr"\sigma"[two heads]b\arr"y"X$ of $x$ such that $y$ is non-degenerate.
\end{lemma}
\begin{proof}
	Consider the set of factorisations of $x$ by morphisms in $\dgm{A}_-$. This is not empty because $(\id_a, x)$ gives one such factorisation.
	Therefore, there exists a factorisation $(\sigma,y)$ whose degree of the domain of $y$ is minimum among this set. It suffices to show $y$ is non-degenerate.
	To this end, suppose $y$ factors as $z\circ\tau$ where $\tau$ is a morphism in $\dgm{A}_-$. Since $(\tau\circ\sigma,z)$ gives a factorisation of $x$,
	the degree of the domain of $z$ is not less than that of $y$. Therefore, $\tau$ must be an identity because $d\colon\dgm{A}_-^\op\arr\omega$ reflects identity.
\end{proof}
\begin{proposition}
	\label{prop:equivalenceElegant}
	For a Reedy category $\dbl{A}$,
	the following are equivalent.
	\begin{enumerate}
		\item %
			\label{item:EZlemma}
			For each presheaf $X\in\Psh[\dgm{A}]$,
			any cell $x\colon a\arr X$
			factors uniquely as $a\arr"\sigma"[two heads]b\arr"y"X$
			where $\sigma$ is in $\dgm{A}_-$ and $y$ is non-degenerate.
%		\item %
%			\label{item:EZlemmaasTerminal}
%			For each cell $x\colon a \arr X$ of any presheaf $X\in\Psh[\dgm{A}]$, the coslice category $x/(\dgm{A}_-/X)$ has a terminal object.
		\item %
			\label{item:strongpushout}
			Any span in $\dgm{A}_-$ admits a pushout in $\dgm{A}$ that is preserved by the yoneda $\dgm{A}\arr[hook]\Psh[\dgm{A}]$.
		\item %
			\label{item:absolutepushout}
			Any span in $\dgm{A}_-$ admits an absolute pushout in $\dgm{A}$.
		\qedhere %
	\end{enumerate}
\end{proposition}
\begin{proof}
	The equivalence of \ref{item:strongpushout} and \ref{item:absolutepushout} follows from a folklore result:
	a small colimit is absolute if and only if it is preserved by the cocompletion.
%	\\\ 
%	[\ref{item:EZlemmaasTerminal}$\Rightarrow$\ref{item:EZlemma}]
%	Let $x\colon a\arr X$ be a cell.
%	The existence of such factorisation is trivial: we can take the terminal object $(\sigma,y)$ of the coslice,
%	and then $y$ is non-degenerate. For the uniqueness, take another factorisation $(\sigma',y')$ of $x$ such that $y'$ is non-degenerate.
%	Since $(\sigma,y)$ is terminal, there is a morphism $\tau\colon(\sigma',y')\arr(\sigma,y)$ in the coslice category.
%	In particular this factors $y'$ as $\cdot\arr"\tau"[two heads]\cdot\arr"y"X$,
%	and hence $\tau$ is an identity.
	\\\ 
	[\ref{item:EZlemma}$\Rightarrow$\ref{item:strongpushout}]
	Suppose \ref{item:EZlemma} and that we are given a span $b\arr"\sigma"[twoheadleftarrow]a\arr"\sigma'"[two heads]b'$ in $\dgm{A}_-$,
	and take a pushout in $\Psh[\dgm{A}]$ as in the diagram below.
	Since the yoneda embedding is fully faithful, it suffices to show the pushout is representable.
	Let us factorise $x$ and $x'$ by \ref{item:EZlemma} as follows so that the cells $y$ and $y'$ are non-degenerate.
	\[
		\begin{tikzcd}[column sep=5mm,row sep=5mm]
			a
			\ar[dd,two heads,"\sigma"']
			\ar[rr,two heads,"\sigma'"]
			\doublecell[rrdd,"\ulcorner"near end]{}
				&
				{}
					&
					b'
					\ar[dd,"x'"]
					\ar[rd,two heads,"\tau'"]
						&
						{}
			\\
			{}
				&
				{}
					&
					{}
						&
						c'
						\ar[ld,"y'"]
			\\
			b
			\ar[rr,"x"']
			\ar[rd,two heads,"\tau"']
				&
				{}
					&
					X
						&
						{}
			\\
			{}
				&
				c
				\ar[ru,"y"']
					&
					{}
						&
						{}
		\end{tikzcd}
	\]
	By the uniqueness of the factorisation of $x\circ\sigma$, we have $(\tau\circ\sigma,y)=(\tau'\circ\sigma',y')$.
	By the universality of the pushout applying to the square $\tau\circ\sigma=\tau'\circ\sigma'$, we obtain a section $u$ of $y=y'$.
	Therefore, we have a factorisation $(u\circ y,y)$ of $y$. Since $y$ is non-degenerate, through factorising $u\circ y$ by the factorisation system,
	we conclude $u\circ y$ is an endo-morphism in $\dgm{A}_+$, which shows $u\circ y=\id_c$ and that $X$ is representable.
	\\\ 
	[\ref{item:strongpushout}$\Rightarrow$\ref{item:EZlemma}]
	By \cref{lem:existenceofNon-deg_factorisation},
	it suffices to show the uniqueness of such factorisation $(\sigma,y)$ for any cell $x$.
	Suppose \ref{item:strongpushout} and that we are given a cell $x$ and its two factorisations $(\sigma,y)$ and $(\sigma',y')$ where $y$ and $y'$ are non-degenerate.
	By taking the pushout of the span formed by $\sigma$ and $\sigma'$,
	we obtain a square $\tau\circ\sigma=\tau'\circ\sigma'$ in $\dbl{A}$ and factorisations
	$(\tau,z)$ and $(\tau',z')$ of $y$ and $y'$ respectively.
	Since the left class of an orthogonal factorisation system must be closed under pushout, $\tau$ and $\tau'$ are in $\dbl{A}_-$.
	Moreover, since $y$ and $y'$ are non-degenerate, $\tau$ and $\tau'$ are identities, and this shows $(\sigma,y)=(\sigma',y')$.
\end{proof}
\begin{definition}
	A Reedy category is \emph{elegant} if the equivalent conditions in \cref{prop:equivalenceElegant} hold.
\end{definition}
\begin{fact}[{\cite[Proposition 4.2]{BR13}}]
	\label{EZReedy_is_elegant}
	A Reedy category is elegant if the following condition holds:
	\begin{itemize}
		\item[(EZ)] %
			Any morphism in $\dgm{A}_-$ is a split epi, and for any parallel morphisms
			$f\colon a\arr[two heads]b$ and $g\colon a\arr[two heads]b$ in $\dgm{A}_-$,
			$f=g$ holds if any morphism $b\arr a$ is a section of $f$ and $g$ simultaneously.
		\qedhere
	\end{itemize}
\end{fact}
\begin{example}
	The category of simplices $\Simp$ is a Reedy category by the (surjection, injection)-factorisation system
	and the ordinary degree function.
	Moreover, this is elegant because $\Simp$ satisfies the condition in \cref{EZReedy_is_elegant}.
\end{example}
\section{FONDIND}
Let $\dgm{A}$ be a Reedy category.
The following definition is after Yuki Maehara.
\begin{definition}
	A presheaf $X$ is \emph{fondind} if any face of a non-degenerate cell is non-degenerate: i.e.,
	$X$ is fondind if
	for each factorisation $b\arr"\iota"[tail]a\arr"x"X$ of a cell $y\colon b \arr X$ with $\iota$ in $\dgm{A}_+$,
	the cell $y$ is non-degenerate whenever $x$ is as well.
\end{definition}
\begin{proposition}
	The class of fondind presheaves is closed under limits and subobjects.
\end{proposition}
\begin{proof}
	A cell in a subobject of a simplicial set $X$ is non-degenerate if and only if it is non-degenerate in $X$.
	On the other hand, a non-degenerate cell $\langle x_i\rangle_{i}\colon a \arr \prod_{i\in I}X_i$ is non-degenerate
	if and only if $x_i$ is non-degenerate for each $i$.
	Those observations immediately show that being fondind is closed under products and subobjects, which shows the proposition.
\end{proof}
\begin{remark}
	The full subcategory of $\Psh[\dgm{A}]$ spanned by fondind objects is a quasivariety in the sense of \cite[16.12]{AHS90}.
	This follows from the above proposition using \cite[Theorem 16.8 and Theorem 16.14]{AHS90}.
\end{remark}
Note that for each presheaf $X$, the category of elements $\dgm{A}/X$ has as its objects cells of $X$.
We write $\ndcat{X}$ for the full subcategory of $\dgm{A}/X$ consisting of all non-degenerate cells.
\begin{lemma}
	\label{lem:connectednondeg}
	Let $X$ be a presheaf.
	Consider the following commutative diagram in $\Psh[\dgm{A}]$.
	\[
		\begin{tikzcd}
			\cdot
			\ar[d,two heads,"\sigma'"']
			\ar[r,two heads,"\sigma"]
				&
				{}
				\ar[r,tail,"i"]
					&
					\cdot
					\ar[dd,"x"]
					\ar[ddll,"f"description]
			\\
			{}
			\ar[d,tail,"i'"']
				&
				{}
					&
					{}
			\\
			\cdot
			\ar[rr,"x'"']
				&
				{}
					&
					X
		\end{tikzcd}
	\]
	Suppose that $x$ is non-degenerate.
	Then we have $\sigma=\sigma'$ and $x\circ i=x' \circ i'$.
\end{lemma}
\begin{proof}
	Considering the factorisation of the morphism $f$ in $\dgm{A}$, we can conclude $f$ is a morphism in $\dgm{A}_+$ since $x$ is non-degenerate.
	Now since both $(\sigma',i')$ and $(\sigma,f\circ i)$ give the $(\dgm{A}_-,\dgm{A}_+)$-factorisation of the same morphism,
	they coincide. This implies $\sigma=\sigma'$ and $x\circ i=x'\circ f\circ i'= x'\circ i'$.
\end{proof}
\begin{proposition}
	\label{prop:ndaf_fondind}
	A presheaf
	$X\in\Psh[\dgm{A}]$ is fondind
	if $\ndcat{X}$ is a final full subcategory of $\dgm{A}/X$.
\end{proposition}
\begin{proof}
	Suppose $\ndcat{X}$ is a final full subcategory of $\dgm{A}/X$.
	To show $X$ is fondind, take a factorisation $a\arr"i"[tail]b\arr"x"X$ of a cell $a\arr"y"X$ by a morphism $i$ in $\dgm{A}_+$
	and a non-degenerate cell $x$.
	We will show $y$ is non-degenerate.
	To this end, consider another factorisation $a\arr"\sigma"[two heads]c\arr"z"X$ of $y$ obtained by \cref{lem:existenceofNon-deg_factorisation}.
	Since $z$ is non-degenerate, it suffices to show $\sigma=\id$.
	Since $\ndcat{X}$ is final in $\dgm{A}/X$, the comma category $y\downarrow \ndcat{X}$ is connected.
	In particular, $(\sigma,z)$ and $(i,x)$ are connected by a zigzag in this comma category.
	Now, iterated use of \cref{lem:connectednondeg} to this zigzag shows $\sigma=\id$.
\end{proof}
\begin{proposition}
	The converse of \cref{prop:ndaf_fondind} is true when $\dgm{A}$ is elegant; i.e.,
	if $\dgm{A}$ is elegant,
	a presheaf $X$ is fondind if and only if $\ndcat{X}$ is final in $\dgm{A}/X$.
\end{proposition}
\begin{proof}
	Suppose we are given a fondind presheaf $X$ and a cell $a\arr"x"X$.
	It suffices to show the comma category $x\downarrow\ndcat{X}$ is connected. Since it is inhabited by \cref{lem:connectednondeg}, it suffices to show that
	any two factorisations of $x$ by non-degenerate cells are connected in this category.
	To this end, take two factorisations $a\arr"f"\arr"y"X$ and $a\arr"f'"\arr"y'"X$ of $x$
	where $y$ and $y'$ are non-degenerate. Moreover, let us decompose $f$ and $f'$ by the factorisatiion system
	as $f=i\circ\sigma$ and $f'=i'\circ\sigma'$.
	Since $X$ is fondind, $y\circ i$ and $y'\circ i'$ are non-degenerate, and hence $(\sigma,y\circ i)$ and $(\sigma',y'\circ i')$ become objects of the comma category.
	Moreover, since $\dgm{A}$ is elegant, those two factorisations coincide,
	and now we have a span $(f,y)\arr"i"[leftarrow](\sigma,y\circ i)=(\sigma',y'\circ i')\arr"i'"(f',y')$ in $x\downarrow\ndcat{X}$.
\end{proof}

\printbibliography
\end{document}
\begin{theorem}
	The following are equivalent.
	\begin{enumerate}
		\item %
			\label{item:Aelegant}
			$\dgm{A}$ is elegant.
		\item %
			\label{item:fondindCharacterisation}
			For each $X\in\Psh[\dgm{A}]$,
			$X$ is fondind if and only if $\ndcat{X}$ is a final full subcategory of $\dgm{A}/X$.
	\end{enumerate}
\end{theorem}
\begin{proof}
	\inred{ongoing}
\end{proof}

\begin{notation}
	Let us write $\Ndaf[\dgm{A}]$ for the subcategory of $\Psh[\dgm{A}]$
	consisting of all presheaves $X$ such that $\ndcat{X}$ is final in $\dgm{A}/X$,
	and morphisms that preserves non-degenerate cells.
\end{notation}
\begin{proposition}
	The inculusion $\Ndaf[\dgm{A}]\arr\Psh[\dgm{A}]$ has a right adjoint.
\end{proposition}
\begin{proof}
	For each presheaf $X$,
	we will show the colimit
	\[
		\overline{X}:=\colim\left(\ndcat{X}\arr[hook]\dgm{A}/X\arr"\dom"\dgm{A}\arr[hook]\Psh[\dgm{A}]\right)
	\]
	gives the cofree object of $X$.
	By definition, there is a canonical morphism $\overline{Y}\arr"\varepsilon_Y"Y$ for each $Y\in\Psh[\dgm{A}]$,
	and this becomes an isomorphism if $Y$ is in $\Ndaf[\dgm{A}]$.
	Suppose we are given $X\in\Psh[\dgm{A}]$ and $Y\in\Ndaf[\dgm{A}]$.
	It suffices to show that any morphism $Y\arr"f"X$ factors through $\varepsilon_X$ uniquely
\end{proof}

